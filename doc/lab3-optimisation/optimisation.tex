% TODO look at code for x2 * 3 * 4 int and double to see what generate code...

\documentclass[a4paper,11pt]{article}
\usepackage{url}
\usepackage{hyperref}
\usepackage{listings}
\usepackage{color}
\usepackage{minted}
\usepackage{tikz}
\usepackage{tabularx}
\usepackage[T1]{fontenc}
\usetikzlibrary{arrows}
\usetikzlibrary{patterns}

\hoffset 0in
\oddsidemargin 0in
\voffset -0.4in
\topmargin 0in
\headheight 12pt
\headsep 0,5in

\marginparwidth 50pt
\marginparsep 5pt
\reversemarginpar


\textwidth 6.5in
\displaywidth 6.5in
\textheight 240mm
\parindent 0mm
\parskip \baselineskip

\newcommand{\code}[1]{\texttt{#1}}

%\newcommand{\heading}[1]{\vspace{2ex}\section*{#1}}
\newcommand{\refsection}[1]{Section \ref{#1}}


\begin{document}

\title{ \bf ENEL464 - Optimisation }
\author{}
\date{}
\maketitle


\section{Introduction}

When writing programs, we use compilers to convert source code into
executables.  This process converts each line of code in an equivalent
statement in assembly.  The basic compilation process will generate a
(usually) correct program, however, it will contain lots of
superfluous instructions that are not really needed to get the correct
output. The clever people who develop our C compilers include an extra
step to the compilation process called \emph{optimisation}. This can
be enabled and used with different levels to try to simplify the
output machine code such that it runs faster (sometimes in the order
of 5--10$\times$ faster!).

\section{GCC}

For GCC specifically, optimisations are controlled using the -O flag:
\begin{minted}{shell}
$ gcc -O3 -o poisson poisson.c
\end{minted}

There are several optimisation levels:

\begin{centering}
    \begin{tabularx}{\linewidth}{ l  X }
        % Flag & Description \\ \hline \hline
        -O0 & optimisations disabled (default) \\ \hline
        -O1 & compiler tries to reduce code size and execution time, without
              performing optimisations that take a great deal of compilation
              time. \\ \hline
        -O2 & perform nearly all supported optimisations that do not involve a
              space-speed tradeoff. \\ \hline
        -O3 & enable more optimisations that may make program bigger (but maintain standards compliance). \\
              \hline
        -Ofast & disregard strict standards compliance in addition to -O3 (i.e.,
                 fast math). \\ \hline
        -Os & -O2 except for optimisations that increase code size. \\ \hline
        -Og & optimise for debugging experience. Equivalent to -O1 except for
              optimisations that make it harder to debug. \\
    \end{tabularx}
\end{centering}

There are hundreds of different optimisations that can be enabled with
specific option flags.  Each optimisation level controls a subset of
these optimisations.  For example, \code{-O3} enables
\code{-funswitch-loops} for loop unswitching.


\section{Compiler Explorer}

There is a fantastic tool called \emph{Compiler Explorer}
(\url{https://godbolt.org/}) which lets you compile code from a variety of
languages (including C/C++) and can show the generated assembly code for
platforms like x86-64, ARM, and so on.

Try compiling the following code with GCC for x86-64:
\begin{minted}{c}
#include <cstdio>

inline int foo (int bar)
{
    int sum = 1;
    for (int i = 2; i < bar; i++)
        sum *= i;

    return sum;
}


void bar()
{
    printf("%i", foo (5));
}
\end{minted}

In this example, the \code{foo} function simply calculates the product of all
the numbers up to the input (i.e., $1 \times 2 \times 3 \times \ldots$). The
generated code looks like:

\begin{minted}{asm}
foo(int):
        push    rbp
        mov     rbp, rsp
        mov     DWORD PTR [rbp-20], edi
        mov     DWORD PTR [rbp-4], 1
        mov     DWORD PTR [rbp-8], 2
        jmp     .L2
.L3:
        mov     eax, DWORD PTR [rbp-4]
        imul    eax, DWORD PTR [rbp-8]
        mov     DWORD PTR [rbp-4], eax
        add     DWORD PTR [rbp-8], 1
.L2:
        mov     eax, DWORD PTR [rbp-8]
        cmp     eax, DWORD PTR [rbp-20]
        jl      .L3
        mov     eax, DWORD PTR [rbp-4]
        pop     rbp
        ret
.LC0:
        .string "%i"
bar():
        push    rbp
        mov     rbp, rsp
        mov     edi, 5
        call    foo(int)
        mov     esi, eax
        mov     edi, OFFSET FLAT:.LC0
        mov     eax, 0
        call    printf
        nop
        pop     rbp
        ret
\end{minted}

Quite a bit going on there for a small function!

If we enable \code{-O1} (add it to the compiler flags), we can see the compiler
detects that the call to \code{foo} is a compile time constant. This lets the
compiler do the math in advance and simply store the result in our program:

\begin{minted}{asm}
.LC0:
        .string "%i"
bar():
        sub     rsp, 8
        mov     esi, 24
        mov     edi, OFFSET FLAT:.LC0
        mov     eax, 0
        call    printf
        add     rsp, 8
        ret
\end{minted}

If we then enable \code{-O3}, we can see the compiler strips out a few
extraneous instructions that are not needed for this program giving us a very
small (and fast) program to run:

\begin{minted}{asm}
.LC0:
    .string "%i"
bar():
    mov     esi, 24
    mov     edi, OFFSET FLAT:.LC0
    xor     eax, eax
    jmp     printf
\end{minted}

\section{Next steps}

You should compare the runtime of your solution in the different optimisation
modes. Does \code{-Ofast} make a difference compared to \code{-O3}? Which
optimisation modes make no change to performance and why?

\begin{itemize}
\item \code{gcc} will output the generated assembly code using the \code{-S} option.

\begin{minted}{shell}
$ gcc -O3 -S -o poisson.s poisson.c
\end{minted}

\item \code{objdump} will disassemble an object file using the \code{-d} option.

\begin{minted}{shell}
$ objdump -d poisson
\end{minted}

It can also annotate the disassembled code with the source code using
the \code{-S} option.

\begin{minted}{shell}
$ objdump -S poisson
\end{minted}

In this case the program needs to be compiled with the \code{-g}
option to include debugging symbols in the object file.  Note,
compiler optimisation code can result in little semblance between the
source and object code.

\end{itemize}

\end{document}
